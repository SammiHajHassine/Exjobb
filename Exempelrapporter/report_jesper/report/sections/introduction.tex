\chapter{Introduction}

\setcounter{section}{1}

%\subsection{Background}

For many years traditional web services have been built upon complex backend systems that serve static HTML-files to the users. When the Web 2.0-era came around, techniques of dynamic content loading such as Asynchronous JavaScript and XML (AJAX) became popular. These techniques allow the application to fetch data from the server without reloading the entire page. However, most of the page is still initially rendered on the server side. What we are seeing today is a change, now the backend acts as a simple API and instead puts a lot more responsibility on the client. The client is often built in Javascript to allow a flexible and interactive user interface. This type of web application, known as Single-Page Application (SPA) or Single-Page Interface (SPI), radically changes the entire architecture. When the number of components in the client grows the complexity is also increased. This makes a good structure or framework a crucial part of the system design, this becomes important since Javascript is an interpreted and loosely typed language which allows many possible faults to be introduced. Today some of the most popular websites are built as SPAs. A few examples are Gmail, Twitter and Foursquare.

%\subsection{Problem description}
%\label{sec:problem_definition_description}

With the traditional approach HTML-files are rendered with the current content of the database. When data is changed the page must be reloaded to force the HTML-files to be re-generated from scratch. Even for sites using AJAX to dynamically change parts of the page this interaction is the most common case. On the other hand, a SPA allows a more flexible and elegant way of dealing with data. Once the user has loaded an initial slim version of the site all the data can be fetched asynchronously. This has several advantages. First an initial page is now rendered faster, since a smaller amount of data is transferred to the user. This is the case since every request doesn't have to include the presentation of the data. It also allows data to only be fetched one single time since the client now can act as a small cache memory of the database. All of this flexibility leads to a more responsive interface and in the end to a better user experience. Unfortunately this flexibility also comes with a price, the frontend is now a lot more complex than before. When data is represented both at the client and at the backend side an interface is needed for the communication in between. This interface needs to be implemented, for purpose of fetching, inserting, updating and deleting data. Problems regarding synchronization between user sessions may also arise since the client now acts as a cache memory to the database. When a user modifies data this may affect other users that are online, i.e. concurrency problems can appear. The responsibility of synchronization between users is now put on the developer. All of this leads to a more complex design of the frontend.

When the frontend becomes bigger and more complex it also requires more memory to run. Since a user often only works with a particular part of a system at a time, it would be a waste of memory to have the entire application instantiated. A structure that allows the developer to decide what parts of the system to run would solve this problem. If the design of these parts of the system could also be re-used in other projects, then the efficiency of the development would be greatly increased.

Another important aspect of software development is maintenance of the code. In bigger projects often several developers are working collaboratively to build the frontend of a system. Code that is easy to read and understand become key aspects to achieve high efficiency and good quality. The use of design patterns becomes an important part of the system design. Unfortunately, it is a complex choice to make since there are many aspects to consider.

Some criticism has been raised against the performance of SPAs in terms of initial loading time \cite{twitter_no_hashbang}. Due to limitations of the HTTP architecture might the initial loading time be higher when using a Javascript frontend compared to the traditional approach. This is the case since the client first downloads the initial page and then executes its Javascript. When this is done the user can be routed to the corresponding page and its content can be loaded. This leads to a higher initial loading time compared to just downloading the plain page from the server. However, once the initial page has been loaded and rendered the Javascript frontend will be able to perform a lot better. Since the initial loading time is a key factor to whether a user is willing to visit the site or not \cite{slow_not_pop}, it has become a crucial aspect of Javascript-based frontends.

Another problem that is introduced when building a frontend in Javascript is if users visit the site and have disabled or cannot run Javascript, then they won't be able to see the content of the site. Even though only 2\% of today's users have Javascript disabled \cite{js_enabled_stats} this still represents a problem. Crawlers from search engines represent for example a small amount of the total number of users but they are still very important to consider. If a crawler cannot see the content of the page it is impossible for it to analyze what the site is all about. This would lead to a poor page rank at the search engine. For many of today's websites this is extremely important.

%\subsection{Goal}

The goal of the thesis is to come up with a system design of a framework for more lightweight single-page applications. The architecture shall be based on design patterns well suited for web frontends. To give a good view of how design patterns can be used in practice the thesis clarifies how they can be typically used in a web application. Based on the result of the system design the framework was implemented, allowing applications to get a high degree of testability as well as encouraging code re-usage between projects. The initial loading time of the page is considered as an important aspect of the framework and it was in all cases minimized. The thesis proposes also a solution to the problem of search engine optimization for single-page applications.

%\subsection{Approach}

The thesis has been carried out at Valtech at their office in Stockholm. Valtech is a global consultancy firm with offices spread all around the world including Europe, Asia and North America. They deliver solutions for the web and mobile market with focus on high quality products and intuitive user interfaces. Their web interfaces often make use of Javascript to achieve a better user experience. Since Valtech's employees have been working with web development for years they have got a lot of experience when it comes to system design of web frontends. By performing interviews, their experience has been as a fundamental source for better understanding what problems are most common when developing Javascript applications.

By looking at design patterns suitable for graphical interfaces an abstract idea of how to form the system architecture was developed. This idea was then further refined to fit a structure that was more suitable for the web. Today there already exists a number of Javascript frameworks, these were analyzed to find different ways of approaching the problems. The majority of the existing frameworks are using techniques that are well known, well tested and have a wide browser support. However, the upcoming HTML5 draft contains new exciting techniques introducing new ways of approaching the problems. These techniques were taken into account when designing the architecture of the framework. During the design phase suitable design patterns were selected for the upcoming implementation of the framework. To be able to design the framework with a high degree of testability a test framework for Javascript was used. It was important to understand how testing typically is carried out and what testing utilities that are commonly used. Once the system design was ready the framework was implemented. To verify that the thesis fulfilled its requirements a simple test application was developed. This application was built upon the framework in Javascript. The purpose was to demonstrate how a developer could use the framework and how the most common problems are solved. Finally the initial loading time and testability were verified by performing a number of tests. The results were then be compared against other competitors.

%\subsection{Scope and limitations}

The thesis is limited to only review design patterns that have a clear connection to frontend development, other design patterns were not considered. Since the thesis focuses on system design of Javascript frontends, aspects regarding the backend were neither included within the scope. The system design of the frontend is not limited to a certain type of backend, any language or design used in the backend shall be able to coop with the frontend as long as they have a common interface for communication. When developing the framework were support for older browsers on the market not taken into account, the framework can rather be used as a showcase of what is possible to do within a near future.

The rest of the thesis is structured as follows:

 \begin{itemize}
	\item Chapter two describes the technical background behind SPAs.
	\item Chapter three discusses design patterns commonly used in graphical interfaces.
	\item Chapter four presents the architecture behind the framework and how search engine optimization problems can be approached.
	\item Chapter five describes the implementation of the framework.
	\item Chapter six presents measurements of the framework regarding testability and initial loading time. It also includes comparisons with other frameworks on the market.
	\item The seventh and last chapter discusses when SPAs are suitable to use and how the future of the framework might look like.

 \end{itemize}
