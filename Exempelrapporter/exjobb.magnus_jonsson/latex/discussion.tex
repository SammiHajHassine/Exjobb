\section {Discussion}

This discussion will be divided into two parts. The first part will discuss my own work, this will reflect on the method used and the difficulties that followed. Any changes from the original plan will also be covered here. The second part will discuss the results and analysis that this thesis delivered. These discussions will lead to new questions and questions that may need more research. Some questions will be be continued in the section of future work where I will propose what could follow this thesis to ensure the usability of systems like PLESpaces.

In the specification of this thesis the main task was to compare a new system like PLESpaces to the existing system, Ping Pong. The question to answer was if this new, proposed system, could be better than the existing one. In the end this turned out to be a question very hard to answer. In what way is “better” defined? Creating a more usable interface will not make a system better if the system itself does not offer the same services as the current one. Investigating the whole system would mean extensive testing, which is far too big for this master thesis. Instead the thesis took an angle to investigate how the new ideas proposed by the ROLE-project could be implemented in PLESpaces while still having a focus on usability.

The thesis specification also specified the method as a usability investigation making use of “various tools in the field of human-computer interactions” and a survey as the main sources of data. The various tools was not defined, which I saw as part of the work, to actually find the most fitting tools for this assignment. These tools were later chosen as a design analysis mainly based on the ten heuristics of usability and a usability analysis using personas and a user survey. The design analysis was pretty straightforward as it is something I usually put my own designs (like websites etc.) through when I develop them. The personas however was something that took more time than I expected. The hardest part was to create the actual personas. At first I wanted to create them in a way that answered the questions I wanted answers to, but I soon noticed this was the wrong way. It was not until I let the personas actually reflect the typical person they were supposed to reflect that they actually took shape and I was finally able to draw conclusions based on them. The personas were a great help but one can argue that they do not cover all user groups who study at the university and I agree. There are so many individual students studying at Uppsala University that personas can not possibly do them justice. Personas should therefore always be used together with real life users.

The real life students are in this thesis represented by the results of a survey. This survey was handed out to students of a distance course during the summer. One can argue that a survey should be distributed to students in more than one course and I agree. If more time and resources were available I would have sent the survey to more students, but sadly there were not. This could suggest that the results are not valid and I would agree that they may not be 100\% correct but they are still useful. First of all the students in this particular course are very used to computers and different web based systems. There is also no guarantee that they are from Uppsala University. Some have not seen a system like this before and may therefore be more open to the new concept. Secondly the answers are, in most questions, quite unanimous making the results usable.

\subsection {Discussing the Results}
The main question was how to implement the psycho-pedagogical integration model into PLESpaces. The results show that it will be very hard to implement the whole method as the ROLE-project suggests. The main difficulties are in phase one and two where the student is expected to define personal goals and find tools to achieve them.

The results from the survey show that phase one is not well accepted. I personally think this is because it is seldom you actually know what you will do in the course. If you are to set personal goals they have to be very specific in order to be achieved. Just a simple “I will improve my English vocabulary” in an English course is difficult to verify and even it is harder to verify if new words learned are a result of the personal goal or if the student would have learned them anyway. Setting specific personal goals would require more knowledge about the course as they would have to be related to the course goals and assignments. Students could of course create the personal goals based on the course goals and assignments if these are available when the course starts. However, I do not think this is the main idea of PPIM and ROLE. The idea, as I have interpreted it, is that students are supposed to take more responsibility for their own learning and learn the things that are not present in course goals and assignments.

Results from both personas and the survey suggests that the second phase also might be difficult to implement. Finding tools may not be the hardest part, but finding the best tool for the specific task may prove difficult. It may even take focus from the task at hand. In my university studies I have always been taught that the most important thing is how you solve a problem, not the exact tools you use. Knowing how to solve a problem but not being able to find the specific tool to solve it will probably cause irritation. The idea of using widgets and letting the students choose their own tools is not a bad idea though, specially in team work. It is always a hassle finding a common platform for the work at hand. PLESpaces could provide that standard platform. My personal suggestion would be to let teachers set up the workspace for the course, as shown in the design sketches, where students are free to add their own widgets if they find any they prefer. Then let students set up their own collaborative workspaces for group assignments. This should, in part, satisfy the ideas of PPIM.

This thesis does not really look that much at phase three and four. Phase three only regards the using the tools to achieve the set goals. This is nothing new from what students do today. Phase four however could maybe be implemented in some minor way. The difficulty in this phase is to give the students time to reflect on their work. The answer isn't to have courses finish two days earlier for reflections as that would lead to time for relaxing, not reflecting. What could be done instead is to merge phase 3 and 4 together. It could be possible to have students reflect on their work while doing it. By simply asking the student if the chosen set of tools is a good set up the student will reflect on the choice. The input can also be used to recommend a set up for next year's course.

If I am to make an educated guess to why this thesis' results turn out the way they do I would have to suggest that the developers of PPIM have not fully investigated the meaning of "Context of use" from the ISO-standard of usability. Whith such a low percentage of acceptance the model has to be reviewed so that the students' routines are taken into account in order to make all the features in the final version of PLESpaces usable to everyone.
