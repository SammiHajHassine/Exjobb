\section {Requirements}
The major stakeholders in this development project are: The ROLE-project, Uppsala Learning Lab and the users. Requirements from the ROLE-project and Uppsala Learning Lab will be handeled in this section while the user's needs will be taken into account in the next section.

\subsection {Requirements from ROLE}
The ROLE-project has stated the requirements for the entire project. This thesis will only list requirements for a ROLE-based platform, such as PLESpaces (Chatterjee, Law,  Verbert. 2009).

\begin {enumerate}
	\item ROLE should provide users with the ability to construct and maintain a Personal Learning Environment. (PLE)
	\item PLE should support the assembly of learning services / tools / resources.
	\item PLE should allow users to switch between parallel learning context.
	\item Sharing and collaborating in learning contexts.
	\item Data based interoperability of services.
	\item Inter-tool communication in the PLE.
	\item ROLE should support the user in transitioning between different learning situations separated by time or organizational barriers / learning contexts. 
	\item ROLE architecture needs to be appealing.
\end {enumerate}

\subsection {Requirements from Uppsala Learning Lab}
The requirements from Uppsala Learning Lab are mainly interpretations and specifications of the requirements from the ROLE-project. There are, however, some additions: One focus area at Uppsala Learning Lab is the idea of separating different courses into specific “spaces”. These spaces are areas where widgets can be added.

\begin {enumerate}
	\item The PLE should consist of widgets.
	\item Widgets should be able to communicate with other widgets.
	\item Each course should have its own widget space in the students' widget area.
	\item Every student should have a personal space along with the course-specific spaces.
	\item Students should be able to create new widget spaces which can be shared with other students.
\end {enumerate}
