\section {Introduction}
Internet today is everywhere. From being limited to a stationary computer connected through a modem hooked up to the regular phone line we can now access the internet from our cellphones, cars, TVs and even refrigerators. We are currently so many users that we are running out of IP addresses for our connected devices. From just being able to send electronic mail and browsing the web, we are now able to book vacations through just a few clicks and download recipes to our mobile phones on the way to the grocery store. A whole new industry has appeared offering new services to meet the demands of today's internet users. Websites like Wikipedia make sure we can always settle the factual arguments that arise in everyday discussions. Twitter allows us to follow whoever we like and get the latest news and thoughts from our chosen sources. There are even sites which offer to send out messages, such as financial information, unspoken secrets and “last word in an argument” to your loved ones when you die.

For the business section the internet has allowed for development of services that facilitate communication and collaboration. The ability to connect to the Internet and interact with people from all over the world has revolutionized the way we work. It is now possible to have meetings with people on another continent without leaving your office. These meetings became even more popular during the non-flight status caused by the Icelandic volcano Eyjafjallajökull's ash cloud in early 2010. Meeting and working on a distance has become more and more popular. Online meetings are also more eco-friendly as there is no need to fly across the ocean to meet your business partner. The ability to connect to your workstation from outside the office has allowed people to work from home a the cabin in the mountains. Cooperation across time zones has also been made more comfortable. By simply sending an email to a colleague in Japan you can receive an answer to your question the next morning without having to pay hefty phone bills or waking someone in the middle of the night. This also allows for continuous work for a project, when the office in Paris has closed it is time for the office in Los Angeles to start their day.

The benefits of the internet has of course made its way into the official sector. In Sweden you can do your taxes over the internet. You can also apply for university studies and loans  to afford your studies. Many universities rely on different IT-based systems to administrate their courses and allow students to download and upload information related to their courses. Some systems more advanced than others.  A popular service is to manage distance courses through web based systems. Some systems have built-in examination methods, where others offer the ability to upload essays which are automatically checked for plagiarism. The universities' systems are designed to lessen the amount of paper work done and allow access to course material at any time. The systems are often rigid and seldom allows students to adapt the environment to personal needs.

The European Commission (EC) funded project “Responsive open learning environments” (ROLE) is a project aimed at creating learning environments where students are given more room to build and personalize their learning environment. The ability to personalize the learning environment is, within ROLE, thought to give students more control over their learning process. This thesis will investigate and give recommendations for improvement of Uppsala Learning Lab's experimental implementation of the testing and development of the ROLE-project technology. The experimental implementation will in this thesis be referred to its project name: “PLESpaces”.

The main idea of this thesis is to make sure PLESpaces and other systems like it are developed with the students' interests in mind.

\subsection {Background}
The four year EC-funded ROLE project started in February 2009 and has focused on web based environments where functionality is included via widgets. Several prototypes have been developed and tested on various learner communities. The studies that have been performed has investigated how learners perceive a small set of technologies offered by ROLE in specific settings. No study has tried to describe the wider set of technologies offered or envisioned within the ROLE project and asked for learners assessment of perceived usefulness with respect to established solutions.

One goal for the ROLE-project is to develop a model for supporting a wide range of learning scenario with strong support for self regulated learning (SRL). The SRL and PPIM concepts will be briefly explained here, for a more detailed explaination, please go to the Previous Work section (section 2.3). The SRL concept divides learning into three phases: the forethought phase (setting goals and planning), the performance phase (actively observe ones own learning) and the self-reflective phase (reflection on ones own work). This concept has then resulted in a  psycho-pedagogical integration model (PPIM) model defined by four steps: \cite{nussbaumer}
\begin {enumerate}
	\item Learner profile model is defined or revised
	\item Learner finds and selects learning resources
	\item Learner works on selected learning resources
	\item Learner reflects and reacts on strategies
\end {enumerate}

Uppsala Learning Lab is a division at Uppsala University with the responsibility to handle learning environments and research new ways of using IT to improve learning \cite{ull}. Within ROLE, Uppsala Learning Lab has a focus on improving user experience and supporting collaboration through widgets as well as finding techniques for improving the quality of widgets.

One critical technique that Uppsala Learning Lab firmly believe will make a difference is to achieve effortless widget communication. This means that widgets should be able to interact with each other and create a work environment that can be adjusted to the user's needs. In order to enable a usable system these widgets and workspaces must be adjusted in such a way that students and teachers will have no problem to create an environment that will help them work efficiently. This widget communication platform, PLESpaces, has been created as an experimental prototype at Uppsala Learning Lab.

\subsection {Problem description}
The people at Uppsala Learning Lab are together with other universities within the ROLE-project developing a system for students and university employees. This system is to be based on the ideas of the psycho-pedagogical integration model developed within the ROLE-project. Uppsala Learning Lab wanted me to perform a usability analysis of the graphical designs which has been made so far and also investigate if and how the psycho-pedagogical integration model can be integrated into their experimental platform, known as PLESpaces.

\subsection {Purpose and Goal}
This thesis is requested by Uppsala Learning Lab to ensure the usability of the new experimental platform. By analyzing the needs of students and the possibilities of a personalized learning environment this thesis will deliver recommendations for PLESpaces.

This thesis will:
\begin {itemize}
	\item Evaluate existing design prototypes developed within the ROLE-project and use them to create interaction design sketches for PLESpaces
	\item Investigate how the psycho-pedagogical integration model can be implemented in PLESpaces in the way intended by the ROLE-project.
	\item List usability recommendations for PLESpaces
\end {itemize}

\subsection {Limitations}
This thesis will deliver recommendations for the development of PLESpaces at Uppsala Learning Lab. These recommendations will only regard usability. All recommendations will be given from the students' point of view.
The thesis will not give any recommendations regarding the software performance or graphical design. The thesis will not include any software development (i.e. coding). 

\subsection {Method}
This section will declare how the goals will be reached. More detailed information on how these methods work and what has been found in research in their fields can be found in the Previous Work section (section 2.4-2.5).

In the field of HCI there are multiple ways of developing software. Common methods are user centered system design, usage centered design and participatory design. This thesis will not use any of these methods directly. Instead, this thesis will use some of the methods taught and discussed at Uppsala University. The main ideas are common for all of the methods above; to create a system that will be easy to use.

In order to reach the goals listed above this thesis must do the following:
\begin {itemize}
	\item Analyze previous designs and look into existing guidelines within the field of interaction design.
	\item Understand which impact the implementation of PPIM may have on the students.
	\item Analyze the needs of the students and investigate how PLESpaces can satisfy them
\end {itemize}

For the first part this thesis will deliver design sketches of the PLESpaces platform. These sketches will be based on previous designs done within the ROLE-project. The existing sketches can be found in the Previous Work section (section 2.5.1).

For the second part this thesis must gather information from the users. This data will be gathered from two sources: One source will be personas and a use case scenario that will represent different student groups in order to cover the needs of students in different fields. The other source will be feedback from actual students taking a distance course at Uppsala University. Due to the limited time and size of this master thesis there will be no data from deep interviews with real life students. Instead all data will be gathered through a user survey.

\subsubsection {Interaction Design}
Throughout the ROLE-project different sketches and ideas of how a widget-based learning system should look have been created by the ROLE-project participants. Starting from these sketches new designs will be created. First the existing setches need to be analyzed. The results from the analysis will then be used to create new sketches. The analysis and creation of sketches will rely heavily on the rules of heuristics created by Jakob Neilsen. These rules are explained in the Previous Work-section (section 2.5). The new sketches will be presented in a user survey to get further feedback from real-life students.

\subsubsection {Usability}
The usability section is composed of two parts. Each having its own starting point. The first part will be done through personas and a use case scenario. This part will reflect how students could act if a system like PLESpaces would be implemented as intended by the ROLE-project. The personas will give data regarding problems that may arise both in individual use and in team work.

The other part, a user survey, will be handed out to real students. These students are taking a distance course during the summer at Uppsala University. The course is called Social Media and Web 2.0. The students use a Learning Management System (LMS) called Ping Pong to communicate and perform certain course assignments \cite{social}. As a distance course during summer there is a good mix of new and experienced university students, some with experience of Ping Pong and some who are completely new to the system. This student group was chosen since their work will mostly be centred around the existing learning management system at Uppsala University. They should therefore be able to give reliable feedback as to which current features are valuable and not valuable.

The survey will investigate how students use the existing system and what they think about it. Do the students use the features included or do they prefer external program to handle certain tasks (e.g. communication)?

The survey will also investigate the students' study behaviour. Where they look for study material, how they communicate, how they cooperate in a group and if they are used to work with many tools open at the same time. This part of the survey aims to answer questions regarding the implementation of the PPIM model into an IT-system.

In the end the evaluation will list usability recommendations for the new experimental platform. The evaluation will also attempt to answer some of the questions asked by Uppsala Learning Lab.

\begin {itemize}
	\item Should students set up their environment from scratch or should the teacher propose a basic version?
	\item Should the student be able to set up his or her own goals?
	\begin {itemize}
		\item Should the teacher suggest goals?
		\item Who should confirm that the goals has been reached – The teacher or the student?
	\end {itemize}
	\item Should the environment offer interactive recommendations?
	\begin {itemize}
		\item Widget / Widget bundle recommendations
		\item Content recommendations
		\item Pedagogical recommendations
	\end {itemize}
\end {itemize}
