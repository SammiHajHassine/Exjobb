\section {Future Work}
These questions are derived from the analysis and the discussion in the previous sections. I would suggest looking into these questions before launching a system like PLESpaces.

\textbf {Interactive Recommendations:} The survey shows that students would like to have contextual  and automatic recommendations inside the system. Students specially requests recommendations as tools to use and where to find reliable sources. This area would need some extra investigation. What kind of recommendations should be shown? When should the recommendations be shown? Should the student be able to turn them off?

\textbf {Making Students use the System:} Today students have their own personal learning environment set up. Making them switch may not be easy, some students may even have paid for commercial program licenses. The students must see a clear advantage in order to change, preferably without being forced into the system. Maybe focus should be on the social aspect of cooperation?

\textbf {Personal Goals:} As it is today students are not very keen on setting their own goals. In fact not a single one in the survey wanted to set their own goals. The questions here is simple: Should the method of setting goals be forced upon the student or not? If yes: How would this be done? If not: Should someone else set up the goals, or should this phase be discarded in the system?

\textbf {Choosing tools:} This phase is not widely accepted. The survey and personas show that students does not want to create their own set up. Instead the teacher should set up a basic version which students can edit at will. Group spaces can be left empty and students build their own workspaces there instead. Is this an acceptable implementation according to the PPIM?
