\appendix
\section {Survey}
The survey was given in Swedish, this is my translation of the survey to English. The survey was done through a web service called kwiksurveys (www.kwiksurveys.com). 

\textbf{Page 1 - Introduction}

A current EU project called ROLE is currently taking place around Europe. The purpose is to encourage students to take more responsibility for their own learning. One step is to allow students to build their own learning environment.
The main quiestion for ROLE is if this could be done with an IT-system. Uppsala Learning Lab who administrate “Studentportalen” and “Ping Pong” are currently developing an experimental platform to investigate this. Please note that there are no current plans to develop such a platform to replace “Studentportalen” or “Ping Pong”. The project at Uppsala Learning Lab is just an experimental platform developed for research.

This survey aims to get students' feedback regarding the current system used at Uppsala University: Which tools are used? Which tools are not used? Which tools are missing?
At the end of the survey you will find a sketch of what the new experimental platform could look like. Following the sketches are a few questions where you are asked what you think of them and if you would use such a system if it was available.
Page 1 – Who are you?

1. Sex

\begin{center}
    \begin{tabular}{ | l | l | }
    \hline
    Male & 57\% \\ \hline
	Female & 43\% \\ \hline
    \end{tabular}
\end{center}

2. Age - Oldest: 38, Youngest: 21, Average: 25

3. I've studied at a university for:

\begin{center}
    \begin{tabular}{ | l | l | }
    \hline
    < 1 year & 0\% \\ \hline
	1-2 years & 24\% \\ \hline
	2-3 years & 29\% \\ \hline
	3-4 years & 14\% \\ \hline
	4-5 years & 19\% \\ \hline
	> 5 years & 14\% \\ \hline
    \end{tabular}
\end{center}

\textbf{Page 2 - Your thoughts on the current system}
At the following questions you can answer from 1 to 5 where 1 = I do not agree at all and 5 = I fully agree.

4. I see myself as an experienced user of Ping Pong:

\begin{center}
    \begin{tabular}{ | l | l | }
    \hline
    1 & 0\% \\ \hline
	2 & 24\% \\ \hline
	3 & 43\% \\ \hline
	4 & 24\% \\ \hline
	5 & 9\% \\ \hline
    \end{tabular}
\end{center}

5. Ping Pong is easy to navigate:

\begin{center}
    \begin{tabular}{ | l | l | }
    \hline
    1 & 0\% \\ \hline
	2 & 9\% \\ \hline
	3 & 29\% \\ \hline
	4 & 48\% \\ \hline
	5 & 14\% \\ \hline
    \end{tabular}
\end{center}

6. I can modify Ping Pong to better suit my needs:

\begin{center}
    \begin{tabular}{ | l | l | }
    \hline
    1 & 9\% \\ \hline
	2 & 43\% \\ \hline
	3 & 34\% \\ \hline
	4 & 9\% \\ \hline
	5 & 5\% \\ \hline
    \end{tabular}
\end{center}

7. I use all the functions available in Ping Pong

\begin{center}
    \begin{tabular}{ | l | l | }
    \hline
    1 & 9\% \\ \hline
	2 & 67\% \\ \hline
	3 & 19\% \\ \hline
	4 & 5\% \\ \hline
	5 & 0\% \\ \hline
    \end{tabular}
\end{center}

8. I have seen the following tools in Ping Pong (even though I have not used them)

\begin{center}
    \begin{tabular}{ | l | l | }
    \hline
    See my progress & 95\% \\ \hline
	Upload/download documents & 100\% \\ \hline
	Frequently asked questions & 95\% \\ \hline
	List students & 100\% \\ \hline
	Bulletin-board & 66\% \\ \hline
	Discussion forum & 100\% \\ \hline
	Contact the teacher(s) & 76\% \\ \hline
	Private messages & 71\% \\ \hline
	Notes & 33\% \\ \hline
	Calculator & 24\% \\ \hline
	Calendar & 24\% \\ \hline
	Document portfolio & 38\% \\ \hline
	Log book & 5\% \\ \hline
    \end{tabular}
\end{center}

					
9. I have used the following tools in Ping Pong

\begin{center}
    \begin{tabular}{ | l | l | }
    \hline
    See my progress & 95\% \\ \hline
	Upload/download documents & 95\% \\ \hline
	Frequently asked questions & 81\% \\ \hline
	List students & 90\% \\ \hline
	Bulletin-board & 43\% \\ \hline
	Discussion forum & 95\% \\ \hline
	Contact the teacher(s) & 52\% \\ \hline
	Private messages & 38\% \\ \hline
	Notes & 5\% \\ \hline
	Calculator & 14\% \\ \hline
	Calendar & 0\% \\ \hline
	Document portfolio & 24\% \\ \hline
	Log book & 0\% \\ \hline
    \end{tabular}
\end{center}

10. Is there any tool you are missing in Ping Pong?


	"Access to material from earlier editions of the course - Students are too limited to the material uploaded by the teacher" \\
	"Having the mail at the same place" \\
	"Possibly some form of simplification and more distinct tabs" \\
	"There should be some sort of time plan (date and task), which are filled as you progress" \\
	"A feed that is constantly present, and in such a way that the feeling of activity is boosted." \\

\textbf {Page 3 – Your learning methods}
Here are some questions regarding your learning methods. You answer on a scale from 1 to 5 where 1 = never and 5 = very often. In some questions you may click “other” and add your own ideas.

11. To communicate with my classmates I use:

\begin{center}
    \begin{tabular}{ | l | l | l | l | l | l |}
    \hline
    Option & 1 & 2 & 3 & 4 & 5 \\ \hline
	Communications inside Ping Pong & 29\% & 5\% & 14\% & 14\% & 38\% \\ \hline
	E-mail & 62\% & 5\% & 5\% & 14\% & 14\% \\ \hline
	Third-party system (MSN, Skype etc) & 62\% & 14\% & 19\% & 5\% & 0\%\\ \hline
	Phone & 67\% & 10\% & 14\% & 0\% & 10\% \\ \hline
	Other & 95\%& 0\% & 0\% & 0\% & 5\% \\ \hline
    \end{tabular}
    Other: Facebook groups.
\end{center}

12. When we are writing a group assignment we:

\begin{center}
    \begin{tabular}{ | p{8cm} | l | l | l | l | l |}
    \hline
    Option & 1 & 2 & 3 & 4 & 5 \\ \hline
	We meet in school or at another physical location & 48\% & 5\% & 5\% & 33\% & 10\% \\ \hline
	We split the task/document into parts and all group members do their part & 29\% & 19\% & 19\% & 14\% & 19\% \\ \hline
	We work in the same document at the same time (e.g. Google Docs) & 48\% & 14\% & 19\% & 14\% & 5\%\\ \hline
	Other & 81\%& 0\% & 5\% & 0\% & 14\% \\ \hline
    \end{tabular}
    Other: Dropbox, Built-in forum.
\end{center}

13. When I work with my own tasks:

\begin{center}
    \begin{tabular}{ | l | l | l | l | l | l |}
    \hline
    Option & 1 & 2 & 3 & 4 & 5 \\ \hline
	I only have one window open. & 90\% & 10\% & 0\% & 0\% & 0\% \\ \hline
	I frequently switch between a few windows. & 29\% & 19\% & 29\% & 10\% & 14\% \\ \hline
	I have a lot of  windows open at the same time. & 5\% & 0\% & 5\% & 19\% & 71\%\\ \hline
    \end{tabular}
\end{center}
	
14. When searching for study material:

\begin{center}
    \begin{tabular}{ | p{8cm} | l | l | l | l | l |}
    \hline
    Option & 1 & 2 & 3 & 4 & 5 \\ \hline
	I visit the library & 43\% & 24\% & 14\% & 10\% & 10\% \\ \hline
	I search for academic articles on the web (Uppsala University's “Samsök”, Google scholar etc). & 5\% & 5\% & 19\% & 19\% & 52\% \\ \hline
	I use free services to search for material (Google, Wikipedia etc) & 5\% & 10\% & 10\% & 24\% & 52\%\\ \hline
	Other & 95\%& 5\% & 0\% & 0\% & 0\% \\ \hline
    \end{tabular}
    Other: Dropbox, Built-in forum.
\end{center}

\textbf {Page 4 – A layout draft for the new experimental platform}

(Students are now shown the images available in the Interaction Design section.)

These images are just a draft of what the system could look like. The imortant part is not the design in itself but the underlying concept.
Look at the images and then answer a few questions using the same pattern as before (1=I do not agree at all, 5 = I totally agree).

The concept is to use a more flexible system than the ones used at Uppsala University today. This should allow the student to create his or her own learning environment.

The platform makes use of small tools (called widgets) which has specific uses. This could be anything from uploading/downloading files, to word processors and communication tools.
Every course has its own space where tools can be added. The course-specific space is created by the teacher. The teacher also add tools that could help the students. Students are then able to add and remove widgets by themselves to create a more personal learning environment. At the bottom you will find a personal area. This area is not course-specific and will follow the user between different spaces.

You can also create new spaces. In these you can add any tools you want. When you create a space you can share it with other students.
In group assignments it could be helpful to create a space just for that assignment and share it with your group members. In that space the group members can work together for example by writing in the same document at the same time.

15. I think the system would be easy to navigate

\begin{center}
    \begin{tabular}{ | l | l | }
    \hline
    1 & 0\% \\ \hline
	2 & 14\% \\ \hline
	3 & 5\% \\ \hline
	4 & 43\% \\ \hline
	5 & 38\% \\ \hline
    \end{tabular}
\end{center}

16. I think the system would be easy to use.

\begin{center}
    \begin{tabular}{ | l | l | }
    \hline
    1 & 0\% \\ \hline
	2 & 5\% \\ \hline
	3 & 24\% \\ \hline
	4 & 33\% \\ \hline
	5 & 38\% \\ \hline
    \end{tabular}
\end{center}

17. At the start of a new course, would you like to create your own space or do you want your teacher to set up a basic version?

\begin{center}
    \begin{tabular}{ | l | l | }
    \hline
    I want to create my own space from scratch & 10\% \\ \hline
	I want suggestions from the teacher. & 90\% \\ \hline
	I do not know & 0\% \\ \hline
    \end{tabular}
\end{center}

18. Do you find it intuitive to open/minimize tools by ticking the checkboxes?

\begin{center}
    \begin{tabular}{ | l | l | }
    \hline
    Yes & 62\% \\ \hline
	No & 14\% \\ \hline
	I do not know & 24\% \\ \hline
    \end{tabular}
\end{center}

19. Tools can communicate with each other. For example, if you click on a youtube link in the chat system or forum the content viewer will load the video directly. This way you do not need to open more windows or tabs. Do you think this would help you in your studies?

\begin{center}
    \begin{tabular}{ | l | l | }
    \hline
    Yes, it would speed up my learning process & 67\% \\ \hline
	I do not think it would help my studies, but it wouldn't hinder them either & 29\% \\ \hline
	No, it would make it more difficult for me to concentrate. & 5\% \\ \hline
	I do not know & 0\% \\ \hline
    \end{tabular}
\end{center}

20. The lower part of the space is reserved for personal tools. These tools will follow you to other spaces as well. Would you use this feature?

\begin{center}
    \begin{tabular}{ | l | l | }
    \hline
    Yes & 62\% \\ \hline
	No & 24\% \\ \hline
	I do not know & 14\% \\ \hline
    \end{tabular}
\end{center}

21. Extra thoughts, criticisms and suggestions.


	"Super idea! I would also like library catalouges and schoolmail and hotmail available" \\
	"Too messy, too much text and boxes" \\
	"I like the idea of a general view. I'm using multiple monitors at once" \\
	"I think the concept can work for a lot of students but not for everyone. I can feel that this kind of work space would give me too much information, causing me to lose focus" \\
	"I think the concept is good but it is important to keep in mind to keep the system as easy as possible to begin working with and understand" \\
	"Very nice! I hope this will be implemented! Design wise I believe in something like the design above, meaning extremely lightweight. Content wise I think it's brilliant!" \\
	"The biggest problem with systems like Ping Pong is that it is too time consuming to upload and share documents. This is where systems like Dropbox are great" \\
	"I like that 'everything' is available at once. In that way I don't have to waste time clicking around to find what I'm looking for. I would like to see the participant-list to be sorted after online status." \\

\textbf {Page 5 – Widget based systems}
Here we will ask you a couple of questions regarding your experience with systems  based on the same idea as the new experimental platform. (1= I do not agree at all, 5=I totally agree) After that a couple of questions regarding setting personal goals will follow.

22. I have worked which consists of small tools – widgets (e.g. Macintosh Dashboard, Google Android, Gadgets in Windows Vista/7, iGoogle)

\begin{center}
    \begin{tabular}{ | l | l | }
    \hline
    1 & 19\% \\ \hline
	2 & 29\% \\ \hline
	3 & 29\% \\ \hline
	4 & 14\% \\ \hline
	5 & 9\% \\ \hline
    \end{tabular}
\end{center}

I believe a system built upon widgets could be made just as easy to use as a traditional systems.

\begin{center}
    \begin{tabular}{ | l | l | }
    \hline
    1 & 5\% \\ \hline
	2 & 9\% \\ \hline
	3 & 24\% \\ \hline
	4 & 43\% \\ \hline
	5 & 19\% \\ \hline
    \end{tabular}
\end{center}

One of the central parts of ROLE is the idea that students should define and follow up on their own goals. These goals will be set outside the formal requirements to pass the course. Achieving the goals will not have an effect on the final grade.

23. Would you like to create your own goals or should your teacher define them?

\begin{center}
    \begin{tabular}{ | l | l | }
    \hline
    I want to create my own goals & 0\% \\ \hline
	My teacher should define my goals & 24\% \\ \hline
	I want create goals together with my teacher & 57\% \\ \hline
	Suggested goals from the teacher which I choose from & 19\% \\ \hline
	I do not know & 0\% \\ \hline
    \end{tabular}
\end{center}

24. Who should decide if the goals have been achieved?

\begin{center}
    \begin{tabular}{ | l | l | }
    \hline
    Me & 10\% \\ \hline
	My teacher & 14\% \\ \hline
	Whoever set the goal (if both student and teacher create the goals) & 67\% \\ \hline
	I do not know & 9\% \\ \hline
    \end{tabular}
\end{center}

25. Do you want the teacher to see if you have reached your goals,

\begin{center}
    \begin{tabular}{ | l | l | }
    \hline
    Yes & 62\% \\ \hline
	Only the goals, not if they have been achieved or not & 24\% \\ \hline
	No & 0\% \\ \hline
	I do not know & 14\% \\ \hline
    \end{tabular}
\end{center}

26. Would you like to receive recommendations from the system on how you should go about achieving your goals?

\begin{center}
    \begin{tabular}{ | l | l | }
    \hline
    Tools that help me in my work & 81\% \\ \hline
	Material (automatic searches) & 81\% \\ \hline
	New sub goals or new similar goals & 67\% \\ \hline
	Students for collaboration & 67\% \\ \hline
	Other & 0\% \\ \hline
    \end{tabular}
\end{center}
