\section {Analysis}
This section will summarize the answers to the questions stated in the introduction (section 1) and some thoughts about them. This section will be a bit more personal than the earlier parts of the thesis as it will reflect the author's interpretations of the results.

The purpose and goal for this thesis were to:
\begin {itemize}
	\item Investigate how the psycho-pedagogical integration model can be implemented in PLESpaces in the way intended by the ROLE-project.
	\item Look at current prototypes and use them to create interaction design sketches for PLESpaces
	\item List usability recommendations for PLESpaces
\end {itemize}

\subsection {Integrating PPIM in PLESpaces}
As stated in the previous work-section (section 2.3) the psycho-pedagogical integration model divides learning into four phases. In the first phase the student is supposed to set up his or her goals for learning. No feedback regarding this is given from personas, but the real life students answering the survey delivered valuable input. Not one of the students answering the survey wanted to create his or her own goals. Instead the teacher should create goals together with the student. This may not be that realistic, specially in large courses. The ability to create goals can be integrated into PLESpaces, but I seriously doubt the possibility would be used by today's students. If we now assume that students will not create their own goals they will probably only use the system to reach the courses' goals, such as assignments. So PLESpaces will in this case be a personal learning environment without focus on the students creating their own goals. One could argue that students already set up sub goals based on the assignments set by the teacher. This is simply a method of problem solving where the main task is divided into subtasks. As a personal reflection I have only taken one course at the university where special emphasis was given to setting up personal goals and following up on them. This course was a 15 credit project course with 22 students and 3 teachers. I doubt this would have worked in a 10 credit math course with 200 students and 5-6 teachers.

In the second phase of PPIM the student is supposed to find tools to reach the goals set in phase one. In the ROLE-project this translates to adding specific widgets to the student's workspaces. Continuing from the analysis of phase one, where students do not create their own goals, students would now add widgets that will help them to solve goals set by the teacher, such as assignments. The information from personas suggest that finding these tools may take focus from solving the actual assignment. Responses in the survey suggest that students want a basic version prepared by the teacher instead of creating their workspace from scratch. This is not that surprising. It is rare, at least in courses I have personally taken that students are expected to find course material on their own. It does happen that we are supposed to find an academic article about something and refer to it. But the basic material is often prepared by the teacher. Again, as a personal reflection I have only been assigned to find material by myself in one or two courses, not counting this thesis.

The third phase regards working with the selected tools to reach the set goals. This phase should be no problem to integrate in PLESpaces as long as the interface is usable. The main issue would be to make students use it in the first place instead of using their own set up with OpenOffice, Firefox, Skype Google Calendar and Hotmail.

The fourth phase is aimed at reflecting on the learning and how well problems were solved. Personally I have rarely done this, at least not consciously. I have noticed that I use my favourite places to find the information I need, but I rarely have the time to actively reflect on what was good and what wasn't after finishing an assignment. The thesis does not give that much feedback on how this could be implemented. As most courses are today I make sure to finish the last assignment by the last day of the course and then I start a new course the next day. Even with an IT-system with support for this there is just no room for these reflections. The one thing that comes to mind is the reuse of tools. If a tool is used in one course and found usable it will likely be used again in later courses and hopefully recommended to other students. This could be seen as a reflection but it would not be a separate phase rather than a merge between phase three and four.

To summarize: PPIM can partially be implemented. The first phase will be difficult. Creating basic goals like assignments will, just as today, make students create sub goals to finish the main goal. The most realistic method would be to encourage students to use a task manager where they list what they need to do. Whether this can be interpreted as setting up goals is something that can be discussed further. The second phase can be partially integrated. I believe it will be very difficult to make students build their environments from scratch. Instead teachers should provide a set of basic workspace to which students can add their own favourite widgets if needed. The third phase should be able to be integrated without problems and the fourth could perhaps be merged into phase three.

\subsection {Design Sketches}

To create the design sketches presented in this thesis I took the most usable parts from the existing sketches and rearranged the workspace to reflect the results I found in my analysis of those sketches. The new sketches are designed for students used to how websites look today. The new designs were presented to the students filling in the survey and the feedback was positive. A big majority thought the system would be easy to use and navigate. Since the work inside the ROLE-project is not finished I doubt the final version will look like this as new features have to be added when new discoveries are made within the project. However, I do believe that some elements in this design will be quite usable in the final design.

\subsection {Usability Recommendations}
The list of recommendations I came up with can be found in the section with the same name (section 5). Here I will just list the most important ones and explain why I find them important.

“Users must see a clear advantage of using the IT system instead of their usual tools.” (section 5.1) - Everyone works differently, we all have our own personal way of solving problems. Some are more effective than others but it is not often possible to say that one way is better than another. This is reflected in the way students work in school as well. When I've taken a look around the classroom during assignments all screens have been different. Information sources are different, the way of writing code is different and the screens show different programs and windows. It will not be easy to convince students that this new system will be better than the system they have set up in their own way at home. It might be a good idea to focus on the social aspects, like communication and cooperation.

“If communication is to take place inside the IT system everyone must have access to it.” (section 5.1) - Communication is arguably one of the most important aspects of team work and needs to be accessible for everyone. I would suggest that direct communication should be kept in the personal area. Having it in the personal area makes sure it is always present. The only issue might be that communication from other courses may interrupt the students work. One way of solving this is to set availability on a course level, so only people in the active workspace can see that the student is online. A widget for forum and general communication within a workspace should not be possible to remove, only minimized.

“Students use the computer as their main source to find relevant sources” (section 5.2) - I think Uppsala Learning Lab should focus on this. If PLESpaces can be used to effectively find sources like articles and books or just to check some fact, like Wikipedia and WolframAlpha, students may find it very useful. Using that feature might be an entry to other widgets as well.

“Automatic recommendations from the system is a good thing” (section 5.2) - It seems students would like to receive recommendations. The most requested recommendations were widgets that might help do their work more efficiently or find relevant sources in a specific subject. From a usability perspective I would suggest that these recommendations are very subtle. I personally find it quite annoying when a program is trying to tell me something that I haven't requested. I'm not sure whether these recommendations should be able to be turned off. I would personally turn them off in one second since the “tip-of-the-day” dialogues in many programs just aren't helpful. These recommendations might be seen as the same thing.

\subsection {Analysis of the Survey Results}
Most of the answers were in line with my personal oppinion as a student at Uppsala university. There were some differences and other notable results I would like to discuss here:

62\% think Ping Pong is easy to navigate: Personally I would have clicked disagree or strongly disagree. This may be becuase I am more used to computers than others so I expect software to be designed and  behave in a certain way. In any case I have found myself clicking every single link in Ping Pong just to find where my teacher uploaded the assignment I am supposed to do.

When asking students what they think of widget communication 5\% answer they think this feature would hinder them. Even though it is very difficult to make a system that everyone will like I find it hard to get the reason for this response. My best guess would be that those who answered that it would hinder them are either not that used to computers or did not really reflect on what the text meant. I believe that the reason behind Microsoft's success is that all programs delivered with the operating system Windows and the office suite Office are very well integrated and communicate with eachother. However, when designing a system with widget communication you could reduce the confusion for users by making sure the user stays in control.


