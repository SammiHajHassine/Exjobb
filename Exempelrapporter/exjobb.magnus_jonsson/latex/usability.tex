\section {Usability Recommendations}
The ROLE-project finds four typical user groups: Users, peers, tutors and teachers. \cite{chatterjee} This thesis will focus on students. This is mainly since there is not enough time to investigate all users. Students are chosen because they are the, by far, largest user group and also because the author of this thesis is a student.

At Uppsala University there are many different courses where students and teachers have different needs and different tools available. In most basic courses the student groups can be large with more than a hundred students attending lectures and writing exams together. Some advanced courses only have 20-30 students in class, allowing the teacher to have a more personal relation to the students. There are also distance courses where the teacher may not even see the students once. Students and teachers in these courses have different needs which must be reflected upon and taken into account when designing a new system.

\subsection {Personas}
This thesis will use personas to represent different user groups. The personas will be presented in the first section. In the second section a scenario will be presented and in the third section the personas will be used in the scenario to find recommendations based on different user groups.

\begin{wrapfigure}{l}{0.2\textwidth}
  \begin{center}
    \includegraphics[width=0.18\textwidth]{Mikael.png}
  \end{center}
\end{wrapfigure}
\textbf {Mikael:} Mikael is 30 years old. Mikael is taking a course in Swedish. He has earlier worked at a film distribution company. After a few years of work he has again found his passion for languages and is now studying to become a translator, hoping to combine his interests in films and languages. Mikael lives with his girlfriend and their 2 year-old daughter. Mikael's girlfriend wants to get married, but Mikael wants to finish his studies and get his career going before planning a wedding.
At the university Mikael is doing fine. He plans his studies and makes sure he has reads everything thoroughly in time for examinations and seminars. He often takes responsibility and sets up sessions for him and his friends to study together.
Mikael is not really a computer guy. He uses his computer to read the news and to update facebook from time to time. But mostly he visits different forums to discuss films in different genres. The favourite genre is Japanese animé. A few years back Mikael spent some time in Japan to get acquainted with the Japanese culture and language. After all the writing on forums and watching foreign films Mikael consider himself to be very proficient in English. His goal is therefore to become a translator for Japanese and English films.

\begin{wrapfigure}{l}{0.2\textwidth}
  \begin{center}
    \includegraphics[width=0.18\textwidth]{Therese.png}
  \end{center}
\end{wrapfigure}
\textbf {Therese:} Therese is 23 years old. She is currently studying the candidate programme in economics. Therese has a fondness for fashion and spends a lot of time visiting and writing blog posts regarding fashion. This led to her getting a job as a salesperson in a fashion store after graduating from high school. Two years later she started her studies at the university. She really liked her work but she felt the fashion business really did not offer a future for her as she did not have a higher education, and designing was not really her calling. Therese is hoping to get a job as a business administrator at a fashion company. 
Therese lives in a student dorm with 5 others and she really likes the social aspect. The only down-side is that the kitchen is always messy. She would take more responsibility to organize cleaning in the dorm, but it is currently managed by one who has lived there longer and the rest does not seem to mind the messy kitchen. 
To stay in shape Therese goes to the gym for spinning and in the summer time she takes long jogging sessions with her favourite music loaded on her iPod. Therese's studies are no problems, the only problem is all the parties going on. As she is active in one of the student nations her social schedule is quite busy with her dorm mates, people from school and also the active people at the nation.

\begin{wrapfigure}{l}{0.2\textwidth}
  \begin{center}
    \includegraphics[width=0.18\textwidth]{Benjamin.png}
  \end{center}
\end{wrapfigure}
\textbf {Benjamin:} Benjamin is 20 years old. After finishing high school he went directly to the university to start a bachelor programme in computer sciences. Ever since his family got their first computer when Benjamin was 5 years old he has been fascinated by computers and computer games. In high school he took a few programming courses, but mostly he learned by himself. It started when Benjamin played computer games and his team needed a website. Benjamin took the opportunity to learn web coding and then moved on to more traditional software development.
Benjamin lives in a student dorm with 11 other people. He likes the people living there but he does not spend that much time with them. He spends most of his free time with his classmates, playing retro computer games.
Computer science studies are quite individual in the beginning which fits Benjamin perfectly. Benjamin does not have a problem with teamwork, he just prefers working by himself and setting up his work in his way. Benjamin's future is not really planned out. He is satisfied with where he is right now and tries to enjoy his studies and everything around it while it lasts. What he will do for real when his studies are finished will have to wait until later.

\subsubsection {Scenario}
The university where Mikael, Therese and Benjamin are studying has a web based IT-system where all courses are managed. Each course has its own web space where teachers can upload information and assignments and where students can upload solutions to the assignments. The three courses that Mikael Therese and Benjamin are taking have been selected for testing the new widget based personal learning environment. This system is based purely on the idea of the psycho-pedagogical integration model. The system is not an implementation of the PLESpaces software.

During the course the students will be given an empty workspace where they are supposed to add widgets they feel useful. They will receive one individual assignment each to finish by using the widgets they added. After the individual assignment they will be given one group assignment each to finish within their groups which will contain random classmates.

\textbf{Mikael} As Mikael is not really a computer guy he has never used any previous system based on widgets. This means that he does not grasp the concept of having small tools that all work independently. Instead he sees the whole workspace as one program that currently does nothing. After a couple of days of trying to understand what the system is about he has managed to add a word processing widget and a widget for finding academic articles. This work has taken much time and Mikael now has to focus on his assignment. Since he is feeling a bit stressed he goes back to his old way of studying, meaning he searches for material in his usual way and writes his assignment in his standard word processor and in the end copies it into the word processing widget and hands it in. The assignment made Mikael quite annoyed, the system just does not work as he expects it to.

For the group assignment he is lucky that someone else in the group has managed to learn a bit more of the system and creates a shared workspace for the group. Mikael is however not alone to have had problems with the individual assignment and there is some resistance against using the new system. In the end the group decides to use the word processing widget for concurrent writing and a shared notepad for writing notes and adding references. For communication the group uses Skype and e-mail.

\textbf {Therese} Therese on the other hand is somewhat used to computers. For the individual assignment she is quick to set up a word processing widget and one widget for spreadsheets. However, since she has never used the system earlier her major problem is that she has no idea what kind of widgets are available. Are there widgets for seeing interest rates in US for the last five years, and what are they called? There are tips in the forums but as Therese has not added the forum widget, in fact she does not even know there is such a widget, so she does not see the them. In the end she has to rely on the tools she usually uses to finish the assignment.

For the group assignment her groups is rather mixed. Some managed to find the forum and read the tips where others never even used the system since they had no understanding of it. Due to time constraints the group work is divided where the members who did not use the IT-system are assigned to find data that can not be found with the current available widgets where the rest, including Therese, use the IT-system to analyse the data and write the report. Since not all of the members are using the system to do their work they are not available for the communication which takes place inside the system.

\textbf {Benjamin} Benjamin's individual assignment is a programming assignment. He is supposed to develop part of program. Students do different parts which each perform a specific task. He starts by looking through the available widgets and finds the widgets for the forum and the one for reading course-related material. He reads through the lecture notes and the assignment specifics. He now has an idea how to finish his assignment, but he needs some more help. Usually he just searches the web for help, so he adds a widget for displaying search results and saves all the links he needs. Lastly he adds a widget for writing code and starts his coding assignment. After a while he realizes that he needs the code on his own computer to compile the program, so he transfers all his code to his own computer and finishes the assignment there, using the IT system as a source for information.

The group assignment is to take the different parts from the individual assignment and merge them together into a whole program. One of the group members takes command and sets up a shared workspace with communication, a widget for reviewing the code all the group members made and a widget for writing the code together. Benjamin is happy that someone else takes command, but is not happy with the way the workspace is set up. He changes the workspace but it changes for everyone else which causes some argument in the group. In the end the group leader again sets up the workspace and tells everyone else to be happy with the layout, or at least stop arguing about it.

\subsubsection {Results from personas}
After looking through the scenario and the personas' problems and success a few concerns arise. These must be taken into account when designing a new IT system like PLESpaces. These will, along with results from the survey be discussed in the analysis and discussion sections.
\begin {itemize}
	\item The concept of widgets and the widget communication may not be clear to everyone.
	\item Users must see a clear advantage of using the IT system instead of their usual tools.
	\item Finding and adding widgets may take focus away from the actual assignment.
	\item Specific widgets must be easy to find and add.
	\item If communication is to take place inside the IT system everyone must have access to it.
	\item Some tasks may not be able to be performed within the IT system.
	\item Shared workspaces may cause confusion if everyone can change it.
\end {itemize}

\subsection {Survey}
A survey was handed out to the students in the course Social Media and Web 2.0 at Uppsala University. The course is a distance course so the surveys could not be handed out personally. Instead it was managed through a free online survey service called Kwiksurveys. To encourage students to fill in the survey a gift card for the cinemas was sent to everyone who finished it. In the end 21 out of 34 students sent in their answers.

The complete survey and results can be found in the appendix (section A). This section will report on the important findings.

The survey starts out by asking a few personal questions, such as age, gender and how long they have been studying. The results show a slight majority of men (57\%) and an average age of 25 with the oldest at 38 and the youngest at 21. The experience from university studies is quite high where everyone has studied at least one year and almost half (48\%) with more than three years of studies.

The survey continues to investigate what the students think of the current IT system, called Ping Pong. The results from the survey shows that the system is easy to navigate (62\% vs 10\%, meaning 62\% agree or strongly agree and 10\% disagree or strongly disagree) but does not allow students to adapt it to their needs (52\% vs 14\%).

The next part concerns how the students study today, in order to adapt a new system like PLESpaces to their needs. The first question regards communications, and it has to be remembered here that the course is a distance course which may affect the answers. The results show that students are no strangers to using a built-in solution, 52\% say they use it often or very often. This can be compared to the other communication options, e-mail (28\%), telephone (10\%) and third party clients, like MSN or Skype (5\%). 
The survey continues by investigating how students study at the computer. When asked if the students are used to multitasking in a IT-environment. 89\% say they often have multiple windows open at the same time when studying. It also finds that 71\% often or very often use the computer to find relevant academic articles and 86\% often or very often use it to find free material (such as Wikipedia articles). Only 20\% often or very often visit the library to find relevant material.

Students are now shown the design sketch from the previous section (section 4.2) and asked for their opinions about it. This sketch is also meant to show the concept of widget based systems.
81\% consider the design to be easy to navigate. 71\% think the design would be easy to use.
This is followed by questions regarding the concept of widget based systems. Students are now asked what they think of widgets communicating with each other. The question asks if this would be assisting or hindering the student's studies. 67\% think it would be helpful and 5\% think it would hinder them. At the bottom of the design there is an area with personal widgets. When asked if the students would use this 62\% said yes and 24\% said no. Regarding the concept of widget based systems 62\% believe a system built on widgets could be made to be at least as easy to use as today's systems. It should also be noted that only 24\% had previous experience in using systems based on widgets.

The survey now moves onto questions specifically asked by ROLE or Uppsala Learning Lab. The first questions asks if the student would prefer to build a widget space from scratch for a course or if the teacher should provide a basic version. 90\% said they would prefer a system suggested by the teacher. Other questions regard the concept of setting up goals for a course. The first questions is who should set up the goals. 57\% say the teacher should set up goals together with the student, 24\% wants the teacher to set up the goals and 19\% would like the teacher to suggest goal from which the student choose his or her personal goals. Notably 0\% wanted to create their own goals.
67\% wants the creator of the goal to decide if the goal is achieved, 10\% say the student should decide and 14\% say the teacher. It should be noted here that students were mostly consistent. Those who wanted the teacher to set the goals also wanted the teacher to decide (or the creator, which in that case would be the teacher). There was one exception, one student wanted the teacher to set the goals but he or she wanted to decide if they were achieved by him/herself.
When asked if the teacher should be allowed to see the goals 62\% said yes, 24\% said the teacher may see the goals, but not if they have been achieved or not. There were no students saying the teacher could not see the goals.
The last questions asks if the student would like to receive recommendations from the system. These recommendations would be based on the goals set. 81\% said they would like recommendations of widgets that could help them in their studies. The same amount would like recommendations of articles or web-based material that could aid them in their studies. 67\% said they would like recommendations of new goals and subgoals. The same amount would also like recommendations of fellow students to cooperate with.
Even though the same amount of students answered yes on the first two, and another amount answered yes on the other two it was not the same students.

\subsubsection {Results from the survey}
The sample group contains an almost equal amount of men and women. The age span is from 21 to 38 where 25 is the mean value. The results could therefore be seen as representative of the students at Uppsala University.

\begin {itemize}
	\item The current system is easy to navigate but does not allow students to personalize it.
	\item Students have no problems using a built-in communication system.
	\item Multitasking and having multiple things on the screen at one time is nothing new.
	\item Students use the computer as their main source to find relevant sources.
	\item Widget communication would probably be a helpful concept.
	\item The concept of widgets would not be difficult to learn.
	\item A personal toolbar would be useful.
	\item A big majority would like to receive a recommended set up for the workspace.
	\item None of the students wants to create his or her goals. Instead the teacher should create them together with the student.
	\item Most students have no problems with teachers seeing their set goals.
	\item Automatic recommendations from the system is a good thing
\end {itemize}
