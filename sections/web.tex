\section{Introduction to Mobile Web Browsing}

Web browsing, both on desktops and mobile phones involves searching for, and viewing web pages that are part of web sites such as “www.bbc.com”. Web browsers, like “Firefox” or “Safari” read web pages and display them to viewing users. How a certain web page is displayed depends on instructions in the underlying HTML (Hypertext Markup Language) code. The HTML code, which in turn is comprised of HTML tags, can provide various types of functionality to a web page, for instance audio playback, video streaming and pictures. However, in order for web pages to appear professional in form of design, and possess more advanced functionality like drop-down menus or sliding images CSS (Cascading Style Sheets) and JavaScript should be used. CSS is code that defines how HTML-tags and HTML-elements are styled and JavaScript is a language that allows web developers to manipulate both CSS- and HTML-code before and after a web page has been loaded.

\subsection{Mobile Web Pages}

Mobile web pages are designed for small mobile phone screens with imprecise input in the form of human fingers, in contrast to desktop-friendly web pages which are designed for computer screens and precise input with a mouse. The huge difference in screen size has enforced web developers to find new ways of approaching the design of web pages if they are intended for mobile phones.

\subsection{Mobile Adaptation of Web Pages}

The reduced screen size and different input on mobile phones compared to desktop computers affects not only the design and display features of web pages, but these factors also affect user interaction. In Gupta et al they describe two main transformations, regarding how a mobile web page is displayed when a user view it from a mobile browser: a linear approach and direct migration. (ref: http://ijcsi.org/papers/IJCSI-8-2-609-613.pdf) With direct migration, the web page will appear in the same way in the mobile phone’s browser as it does on a desktop computer; no transformation to the desktop web page is performed. Consequently, text, images, links and other web elements will be very small on the screen. Because the elements will substantially small, text will be hard to read, images hard to interpret and it will be difficult to click on links since human fingers will be relatively larger than the links themselves. Action taken by users will most certainly be to zoom in on the page in order to interact with it. This can cause feelings of frustration during interaction (HITTA KÄLLA), and feelings of aversion towards the page. In summary, mobile web pages should be adapted to the features and specifics of mobile phones in order for them to be more easy and pleasant to view and interact with.
With the linear approach the page is transformed (redesigned), to match the reduced screen size of mobile phones in such a way that web page areas (areas with different web elements such as text or images) are presented after each other in a long linear list. The linear list is displayed as a single column to fit easily inside small screen of the mobile phone. With this approach it is not required that users scroll horizontally or zoom in and out to access the available information. Instead they only need to either to scroll down vertically, or click on links which suits the way that people interact with their mobile phones.

\subsubsection{Difficulties}

Incentives to adapt mobile web pages to the features of mobile phones exist, but the adaptation can be rather challenging. The reduction in screen size is extensive and introduces limitations on how much information that can be displayed at once. A lot of web pages have rich content that do not fit all at once on a mobile device, and as Gupta et al (ref: http://ijcsi.org/papers/IJCSI-8-2-609-613.pdf) mentions an analysis have to be made according to the specifications of the devices. Decisions on how the desktop-based web page content shall be prioritized and presented on a mobile device have to be made, and the design process that follows can be complicated. Seeholzer and Salem, Jr (ref: http://crl.acrl.org/content/72/1/9.full.pdf+html) indicated through user study related to mobile websites that the design process of websites presented on a mobile device can be a difficult task. The reason was the risks of achieving mobile websites with pared-down features. Some of the participants the study stated that when they access pared-down mobile websites they did not feel that they were on the internet, and that they looked for a more dynamic experience. It is obvious that the amount of information presented on the reduced screen of a mobile device will be significantly smaller, but what might not be as clear is the decrease in the functionality or dynamic features which can be of importance.

