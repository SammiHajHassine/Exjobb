\section{Introduction}

Web development is a field that rapidly evolves, and web developers continuously need to keep themselves updated of new techniques and trends regarding the subject. The visual design and functionality of websites has gone through a great change since they were introduced, and the number of users on the Internet has increased immensely. Almost every company or organization has a website, and an excessive amount purchases are being done through Internet based services. Internet and website browsing is an important part of the contemporary lifestyle, which stresses the need for efficient and proficient techniques concerning web development. 
There are an enormous number of websites on the Internet, why measurements are needed in order to attract visitors. It is probably not enough to merely develop a website and promote it through various marketing channels in order for it to become successful. Websites should preferably be easy to use, visually pleasant and more depending on the indented audience. Furthermore, people today largely use other devices than desktop computers to browse on websites, such as smartphones and tablets, which introduces new challenges regarding web development.


\subsection{Reading Instructions}

urthr5hdrhrtd

\subsection{Background}

When developing websites and web applications today, one of the challenges is that the result not only works, and is useful on a computer, but consideration must also be given to the fact that users also extensively browse websites on mobile phones. With the introduction of smartphones such as “iPhone”, which contains a built in web browser, the mobile browsing of websites was introduced and developers started talking about mobile webpages. Mobile webpages are websites that are designed for, and easy to use when browsing from a mobile device. This type of design, amongst other measurements, includes a reduction of the material on a desktop based website in order for the user to more easily perceive and navigate through the information. However, a great deal of websites are not available in a mobile format, and when user browse these sort of webpages they are required to zoom in and out frequently in order to push buttons, follow links or read text because these type of web elements are too small when not customized for the mobile format. 
There exist different techniques for developing mobile suitable websites. One way to express a desktop website in a mobile browser is to create a new webpage and style it according to the size of mobile phone screens. Another way to implement a mobile suitable website based on a desktop website is to apply the principles of responsive design. Unlike the solution mentioned where two separate websites are created for desktops and mobile devices respectively, responsive design aims to extend the desktop site with mobile functionality. This function in such a way that the desktop webpage automatically changes layout and design (in accordance with how the mobile webpage is supposed to look like) when browsed from a mobile device.
According to “StatCounter Global Stats”, Internet access through mobile devices has gone from 0.7\% in January 2009, to 8.5\% in January 2012 (cite). The use of mobile devices as a mean for internet browsing has increased rapidly, which might be an indication that effort should be put into the development of mobile webpages. However, mobile web development can be rather complex in terms of information presentation. The substantial decrease in screen size when going from a desktop to a mobile device complicates the process of providing a general overview of all the information on a website.



\subsection{Problem}

\subsection{Purpose and Goal}

\subsection{Restrictions}

\subsection{Method}
