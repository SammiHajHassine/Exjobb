\section{Introduction}

Web development is a field that rapidly evolves, and web developers continuously need to keep themselves updated of new techniques and trends regarding the subject. The visual design and functionality of websites has gone through a great change since the early beginning of webpages, and the number of users on the Internet has increased immensely. Most companies, or organizations today, have a website, and various types of Internet services provide a platform for online purchases. Website browsing is an essential part of the contemporary lifestyle, which stresses the need for efficient and proficient techniques concerning web development. 

There are an enormous number of websites on the Internet which suggests the need of measurements to attract visitors. It is probably not enough to develop a website and promote it through various marketing channels in order for it to become successful. Websites should preferably be easy to use, visually pleasant and more depending on the indented audience. Furthermore, people today largely use other devices than desktop computers to browse on websites, such as smartphones and tablets, which introduces new challenges regarding web development.



\subsection{Reading Instructions}

urthr5hdrhrtd

\subsection{Background}

When developing websites and web applications today, one of the challenges is that the result not only works, and is useful on a computer, but attention must also be given to the fact that users also extensively browse websites on mobile phones. With the introduction of smartphones such as “iPhone”, which contains a built in web browser, mobile browsing of websites increased, and web developers started talking mobile webpages. Mobile webpages are designed for, and easy to use when browsing from a mobile device. This mobile design pattern includes a reduction of the data on a desktop based website in order for users to more clearly understand and navigate through the information. However, a great deal of websites are not available in a mobile format,  when users browse to these webpages that  on a mobile device  they often need to zoom in and out in order to click on links, read text or get an overview because most web elements are too small when not customized for the mobile format. 

There exist different techniques for developing website adapted for mobile devices. One way to express a desktop website in a mobile browser is to create a new webpage and design it according to the size of mobile phone screens. Another way to achieve a mobile suitable website based on a desktop website is to use the principles of responsive design. Unlike the solution mentioned where there exist two different websites for desktops and mobile devices respectively, responsive design aims to enhance the desktop site with mobile functionality. Responsive design function in such a way that the desktop webpage automatically changes layout and design (in accordance with the intended layout of the mobile webpage) when browsed from a mobile device.

According to “StatCounter Global Stats”, Internet access through mobile devices has gone from 0.7\% in January 2009, to 8.5\% in January 2012 (cite). The use of mobile devices as a mean for internet browsing has increased rapidly, which might be an indication that effort should be put into the development of mobile webpages. However, mobile web development can be quite difficult in terms of information presentation. The substantial reduction in screen size when going from a desktop to a mobile device complicates the process of providing a comprehensive overview of all the information on a website.




\subsection{Problem Description}
In the field of mobile web development the issue regarding presentation of information often arises. The information presented could ideally be structured in a way that enhances the experience of the visiting users. Hence, the layout and structure of a website’s information can be crucial to its success. This issue can clearly aggravate in relation to the amount of information presented. 

Another problem that is in direct relation to the layout of information is how to display hierarchal information. Information structured in a hierarchal manner on a website takes shape in such a way that all information available on the site cannot be viewed at once, and in order to view certain information navigation through the hierarchy is necessary, i.e. clicking on certain web elements, such as links or buttons, in a relatively predetermined order. One of the essential parts of this issue is the configuration and design of the navigation through the hierarchy because if it is overcomplicated to find certain information, users will probably be frustrated and annoyed. 

It can be an even greater issue to present hierarchal information on mobile websites in comparison to a desktop websites. Websites presented on a computer screen can display more content at once and provide a more comprehensive overview of the information than on a mobile device. Furthermore, the complexity of the information presentation is enlarged by the fact that mobile browsing is performed with the help of users’ fingers, unlike computers that are being controlled by a mouse. In addition, users’ fingers, in many cases, relatively large in relation to size of the screen on a mobile phone. Therefore, solutions to the problem where as much information as possible is being presented on the phone's reasonably small screen, will probably fail in a user experience point of view. The challenge lies in finding ways to present hierarchal information with its belonging navigation structure that are efficient and intuitive, and which does not affect the user experience in a negative way. 

The problem that this thesis aims to analyze is the presentation of websites containing hierarchal information with a navigational structure in a mobile interface. A user experience point of view will form the basis for the analysis.

\subsection{Purpose and Goal}
TODO
\subsection{Limitations}
The concept of mobile devices will in this thesis only cover the mobile phone. Another system that can be categorized as a mobile device is the "Tablet", such as the "iPad". An investigation on the presentation of websites with hierarchal information on tablets could be of interest as well, but the screen size of tablets does not differ as much from on a desktop computer in comparison with the mobile phone. Furthermore, it is usually not as a significant problem in the development of these types of websites on tablets.

This thesis will also be limited in the number of investigated solutions. There exist a variety of solutions to the problem that this thesis aims to analyze, but due to a limited time span the maximum number of investigated solutions will be 2 or 3.

\subsection{Method}
TODO